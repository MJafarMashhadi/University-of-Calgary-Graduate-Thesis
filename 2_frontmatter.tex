% Sample University of Calgary Thesis
% This file contains the FRONT MATTER other than the title page

\chapter{Abstract}
% \begin{abstract}
% \section{abstract}
Inferring behavior model of a running software system is quite useful for several automated software engineering tasks, such as program comprehension, anomaly detection, and testing. Most existing dynamic model inference techniques are white-box, i.e., they require source code to be instrumented to get run-time traces. However, in many systems, instrumenting the entire source code is not possible (e.g., when using black-box third-party libraries) or might be very costly. %One useful scenario for automated black-box behavior inference is in software control units (such as autopilots), where the software system's reactions over time changes based on the inputs. Run-time state models of such systems are very powerful means for anomaly detection and debugging. 
Unfortunately, most black-box techniques that detect states over time are either univariate, or make assumptions on the data distribution, or have limited power for learning over a long period of past behavior. 
To overcome the above issues, in this thesis, I proposed a hybrid deep neural network that accepts as input a set of time series, one per input/output signal of the system, and applies a set of convolutional and recurrent layers to learn the non-linear correlations between signals and the patterns, over time. 
I have applied this approach on two real UAV autopilot case studies: one from our industry partner, MicroPilot (MP in short), with half a million lines of C code, and one widely used open source solution: Paparazzi. 
I ran more than 1200 system-level tests in total (to generate the input data) and inferred the system's internal state, over time.
In case of Paparazzi, as it did not include system tests like MP, I created a tool that generates and executes meaningful test scenarios.
Comparison with several traditional time series change point detection techniques showed that this approach improves their performance by up to 102\% in MP's case and 49\% in Paparazzi's, in terms of finding state change points, measured by F1 score. I also showed that this state classification algorithm provides on average 90.45\% F1 score for MP and 79.43\% for Paparazzi, which improves traditional classification algorithms by up to 17\% in MP's case and 15\% in Paparazzi's.

%Furthermore, I verified efficacy of this approach with a second case study: Paparazzi open source autopilot. As it did not include system tests like MP, I created a tool that generates and runs fuzzy test scenarios. Using the tool I generated and executed 378 test cases %325 test cases using different permutations of a flight plan; I used the fuzzy test execution tool to run 378 simulations using those scenarios; some scenarios were used twice but due to fuzzing in other parameters the generated data are different. I used the data that was produced during those executions for replication. 
%The improvements of this approach over the baselines, as measured by F1 score, were up to 49\% in change point detection and up to 68\% for state detection.
In addition, by creating a hyper-parameter tuning pipeline using grid search technique, despite having a way smaller training set in the second case study (7 times smaller compared to the first one), I managed to get a better performance, up to 37\% better, out of the neural network model as measured by 8 metrics.
The tuning performance is compared to using the same hyper-parameters that worked for MP's case, for Paparazzi.%  As a result, in addition to confirming that the technique is applicable on other software, I.

% \end{abstract}
% \keywords{Recurrent Neural Network; Convolutional Neural Network; Deep Learning; Specification Mining; Black-box Model Inference; Time series;}



\chapter{Acknowledgments}
I would like to appreciate the support from my graduate supervisor, Dr. Hemmati, and thank him for his support and smart and helpful advice over the course of my studies.
I cannot forget to thank Afrouz, for being there for me and keeping my morales up at all times; Also for helping me proofread this thesis.
Dr. Walkinshaw and Dr. Westwick for their input in parts of my research.
The Moses, for helping me with make the charts and visualizations more appealing.
Hamed, for sharing his knowledge of Machine Learning, recommending me useful talks, books, papers, etc.
Adam, for his thorough answers to my questions about MP's autopilot.
MicroPilot Inc. and NSERC for providing financial support for this research.
The anonymous reviewers of the papers I published for their constructive comments.
And finally, to my family and all the friends who supported me and made it a more pleasant experience.


\dedication{To the ones who made this a smoother journey.} 


\tableofcontents

% If you have no tables, delete the next line

\listoftables

% If you have no figures, delete the next line

\listoffigures

% Consult Ch 9 of the memoir class manual on how to set up other
% content lists. Note that memoir does not automatically clear the
% page for these. ucalgmthesis fixes this for the default table of
% contents and lists of tables and figures, but not for anything you
% define

\chapter{Contributions}
During this research, I have made contributions in form of publishing papers, datasets, contributing to the open source, etc. 
Since not all of them ended up in this thesis, I would like to quickly enumerate them so they will not go unnoticed:
\begin{enumerate}
    \item ICPC '19 conference paper: An empirical study on practicality of specification mining algorithms on a real-world application \cite{mashhadi2019empirical}
    \item IST journal paper: Interactive semi-automated specification mining for debugging: An experience report \cite{jafar2019interactive}
    \item ASE '19: presenting the above journal paper in journal-first-conference-second track
    \item Salvaged a Java instrumentation tool that is used for white-box analysis: \url{https://github.com/sea-lab/JInstrumenter}
    \item Developed a fuzz test generator/executor/flight data recorder tool specialized for Paparazzi: \url{https://github.com/MJafarMashhadi/pprz_tester}
    \item ASE '20 conference paper: Hybrid Deep Neural Networks to Infer State Models of Black-Box Systems
    \item Contributed ~10 pull requests to Paparazzi during the development of the fuzz testing tool. They included adding new features, patches for fixing bugs, and adding to the documentations. My changes are included in Paparazzi's latest release.
\end{enumerate}
