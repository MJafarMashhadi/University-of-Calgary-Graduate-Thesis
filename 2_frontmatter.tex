% Sample University of Calgary Thesis
% This file contains the FRONT MATTER other thna the title page

\chapter{Abstract}
% \begin{abstract}
% \section{abstract}
Inferring behavior model of a running software system is quite useful for several automated software engineering tasks, such as program comprehension, anomaly detection, and testing. Most existing dynamic model inference techniques are white-box, i.e., they require source code to be instrumented to get run-time traces. However, in many systems, instrumenting the entire source code is not possible (e.g., when using black-box third-party libraries) or might be very costly. %One useful scenario for automated black-box behavior inference is in software control units (such as auto-pilots), where the software system's reactions over time changes based on the inputs. Run-time state models of such systems are very powerful means for anomaly detection and debugging. 
Unfortunately, most black-box techniques that detect states over time are either univariate, or make assumptions on the data distribution, or have limited power for learning over a long period of past behavior. 
To overcome the above issues, in this thesis, I proposed a hybrid deep neural network that accepts as input a set of time series, one per input/output signal of the system, and applies a set of convolutional and recurrent layers to learn the non-linear correlations between signals and the patterns, over time. 
I have applied this approach on a real UAV auto-pilot solution from our industry partner with half a million lines of C code. 
I ran 888 recent system-level test cases and inferred states, over time. 
Comparison with several traditional time series change point detection techniques showed that this approach improves their performance by up to 102\%, in terms of finding state change points, measured by F1 score. I also showed that this state classification algorithm provides on average 90.45\% F1 score, which improves traditional classification algorithms by up to 17\%.

\hl{Furthermore, I verified efficacy of this approach with a second case study: Paparazzi open source auto-pilot.} As it did not include system test like MP, I created a tool that generates and runs fuzzy test scenarios. I used the data from 378 simulations for replication. The improvements of this approach over the baselines, as measured by F1 score, for change point detection was up to 49\% and for state detection was up to 68\%.
In addition, by creating a hyper-parameter tuning pipeline using grid search technique, I managed to get a better performance out of the neural network model as measured by 8 metrics (compared to using the same hyper-parameters that worked for MP's case, for Paparazzi).%  As a result, in addition to confirming that the technique is applicable on other software, I.

% \end{abstract}
% \keywords{Recurrent Neural Network; Convolutional Neural Network; Deep Learning; Specification Mining; Black-box Model Inference; Time series;}



\chapter{Preface}

This thesis is an original work by the author. 
% \begin{itemize}
%     \item The first chapter frames the problem being tackled. 
%     \item The second chapter explains the data collection process that is step 0 in performing any analysis on the software. This chapter gets more technical and points to my open source contributions and the testing tool set I developed.% for data collection purpose. 
%     The details are moved to an appendix.
%     \item Chapter~3 Contains the manuscript I published in ASE '20 conference. In that paper, I introduce a deep neural network for the task and compare it to several baselines. The case study in this paper is the auto-pilot developed by our industry partner.
%     \item The next chapter is a continuation of chapter~3. I replicated the same study on another auto-pilot software, an open source one. In addition to that, I used a hyper-parameter tuning technique to improve the results even more.
%     \item The last chapter wraps up the results and introduces some lines for future work.
% \end{itemize}


\dedication{To the ones who made this a smoother journey.} 


\tableofcontents

% If you have no tables, delete the next line

\listoftables

% If you have no figures, delete the next line

\listoffigures

% Consult Ch 9 of the memoir class manual on how to set up other
% content lists. Note that memoir does not automatically clear the
% page for these. ucalgmthesis fixes this for the default table of
% contents and lists of tables and figures, but not for anything you
% define
