% Sample University of Calgary Thesis
% This file contains the FRONT MATTER other than the title page

\chapter{Abstract}
% \begin{abstract}
% \section{abstract}
Inferring behavior model of a running software system is quite useful for several automated software engineering tasks, such as program comprehension, anomaly detection, and testing. Most existing dynamic model inference techniques are white-box, i.e., they require source code to be instrumented to get run-time traces. However, in many systems, instrumenting the entire source code is not possible (e.g., when using black-box third-party libraries) or might be very costly. %One useful scenario for automated black-box behavior inference is in software control units (such as autopilots), where the software system's reactions over time changes based on the inputs. Run-time state models of such systems are very powerful means for anomaly detection and debugging. 
Unfortunately, most black-box techniques that detect states over time are either univariate, or make assumptions on the data distribution, or have limited power for learning over a long period of past behavior. 
To overcome the above issues, in this thesis, I proposed a hybrid deep neural network that accepts as input a set of time series, one per input/output signal of the system, and applies a set of convolutional and recurrent layers to learn the non-linear correlations between signals and the patterns, over time. 
I have applied this approach on a real UAV autopilot solution from our industry partner with half a million lines of C code. 
I ran 888 recent system-level test cases and inferred states, over time. 
Comparison with several traditional time series change point detection techniques showed that this approach improves their performance by up to 102\%, in terms of finding state change points, measured by F1 score. I also showed that this state classification algorithm provides on average 90.45\% F1 score, which improves traditional classification algorithms by up to 17\%.

\hl{Furthermore, I verified efficacy of this approach with a second case study: Paparazzi open source autopilot. As it did not include system tests like MP, I created a tool that generates and runs fuzzy test scenarios. Using the tool I generated and executed 378 test cases %325 test cases using different permutations of a flight plan; I used the fuzzy test execution tool to run 378 simulations using those scenarios; some scenarios were used twice but due to fuzzing in other parameters the generated data are different. I used the data that was produced during those executions for replication. 
The improvements of this approach over the baselines, as measured by F1 score, were up to 49\% in change point detection and up to 68\% for state detection.
In addition, by creating a hyper-parameter tuning pipeline using grid search technique, despite having a way smaller training set compared to MicroPilot's, in the second case study I managed to get a better performance out of the neural network model as measured by 8 metrics. The tuning performance is compared to using the same hyper-parameters that worked for MP's case, for Paparazzi.}%  As a result, in addition to confirming that the technique is applicable on other software, I.

% \end{abstract}
% \keywords{Recurrent Neural Network; Convolutional Neural Network; Deep Learning; Specification Mining; Black-box Model Inference; Time series;}



\chapter{Acknowledgments}
I would like to thank my graduate supervisor, Dr. Hemmati, for his support and helpful advice over these years. 



\dedication{To the ones who made this a smoother journey.} 


\tableofcontents

% If you have no tables, delete the next line

\listoftables

% If you have no figures, delete the next line

\listoffigures

% Consult Ch 9 of the memoir class manual on how to set up other
% content lists. Note that memoir does not automatically clear the
% page for these. ucalgmthesis fixes this for the default table of
% contents and lists of tables and figures, but not for anything you
% define
